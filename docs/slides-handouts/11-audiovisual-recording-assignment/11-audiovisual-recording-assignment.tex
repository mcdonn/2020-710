\documentclass{article}

\usepackage[margin=1in]{geometry}
\usepackage{mdframed}
\newenvironment{myenv}[1]
  {\mdfsetup{
    frametitle={\colorbox{white}{\space#1\space}},
    innertopmargin=10pt,
    frametitleaboveskip=-\ht\strutbox,
    frametitlealignment=\center
    }
  \begin{mdframed}%!TEX encoding = UTF-8 Unicode
  }
  {\end{mdframed}}

\usepackage{fancyhdr}
\pagestyle{fancy}
\fancyhf{}
\rhead{3/10/2020}
\lhead{LING710}
\chead{}
\rfoot{Page \thepage}

%\pagestyle{firstpage}
%\fancyhf{}
% \textsc{Department of Linguistics}\vspace{0.75in}}}

\usepackage{enumitem}

\title{Metadata}
\date{}
%\pagenumbering{gobble}
\begin{document}\thispagestyle{empty}
%\maketitle
\begin{center}
{\huge\textbf{Audiovisual recording \& transcription}}\\\smallskip
{\large \underline{Due: 3/31/2020}}
\end{center}

\begin{myenv}{Objectives}
In this assignment, groups of 2-4 students will make a 50+ minute recording of conversation between multiple participants and collect important information about the recording event. Then, each student selects a different three minute segment from the recording that they will transcribe in the coming weeks.\\
\end{myenv}

\section{Record a spoken interaction}
Record a spoken interaction. Your recorded interaction should have the following attributes:

\begin{itemize}
  \item a naturally occurring interaction (not too staged or contrived),
  \item 3 to 5 speakers (preferably),
  \item contains at least some lively back-and-forth interaction,
  \item contains some overlap, but not too much (approximately a tenth to a fourth of the lines being overlapped),
  \item lasts at least 50 minutes (whether speakers talk the whole time or not),
  \item is in English (let's talk if this is a hardship),
  \item excellent video and audio quality with good lighting and framing, little background noise, recorded on high quality equipment.
\end{itemize}

\noindent Because of the limitations on the video equipment, you will need to work together with one or other students in the class. Since you will be working quite intensely with this recording, be sure it is a good one! If your first attempt fails, leave time to do a second recording. You may need to be part of the conversation.\\

\noindent If possible, you should try to capture the entire natural speech event from beginning to end in your recording. Ideally the conversation will turn out to be a lively one, to provide adequate challenge in transcribing. The minimum of three speakers is intended to increase the likelihood of lively interaction and turn-taking, including stretches with overlapping speech. The limit of not more than five speakers is to avoid too much overlap, in the interest of feasibility---a reasonably easy transcription.\\

\section{Collect metadata and other contextual information}

Each group should collect metadata about the event and speakers. At the time of the recording or immediately afterward, each group must also collect important information about the recording event and produce the following items:

\begin{description}
  \item[Metadata] Fill in the spreadsheet \texttt{audiovisual-recording-metadata} in the folder on Google Drive \texttt{11\--audiovisual\--recording\--assignment}.
  \item[Document the setting] Take photographs of the recording setup, including the placement of the equipment and speakers. You may also draw a setting diagram that maps the setting in which the recorded event took place. Indicate the relative locations of relevant aspects of the setting, such as the participants, microphones, table or other salient furnishings, entrance/exit, etc. 
  \item[Event description] Write an event description in 1-2 paragraphs that describes the recorded event in your own words.
%  \item Decide on names for your speakers (if they want pseudonyms) and put these---or real names for participants who want you to use their real names---into the ``Voice Identification Table'' document.
\end{description}

%\begin{center}
%710-YYYYMMDD-G\#.mp4
%710-YYYYMMDD-G\#.wav
%\end{center}

% (In other words, if I made a recording on February 20, 2017, I would name the file: 710-20170220-McDonnell.wav). Upload the entire recording Google Drive.

\section{Choose a segment for transcription}
Each person should select a unique three-minute segment of the longer recording for transcription. Listen to the recording, and decide within your groups which three minute portion each person will transcribe---you will use this selection to learn transcription. Each segment should have a fair amount of overlap (approx. 25\% of the lines overlapping), with lively back-and-forth conversation and little background noise. Be sure you can understand what's going on in the two minute segment.

%Once you have chosen a segment for transcription, export the selection to WAV format (demonstrated in class). TAKE CARE NOT TO ALTER YOUR ORIGINAL RECORDING. Give this new exported segment the following filename:

%710-YYYYMMDD-G\#-Yourlastname-segment.wav
%(for example: Berez-segment.wav)

%Upload the segment on Google Drive.

%Complete the Voice Identification Table: listen to the recording and transcribe the first line or two from each speaker, making note of the time the words are uttered. This will help others identify voices in your recording.


\begin{myenv}{Materials to be submitted}
\noindent Please submit recordings and associated files in a single folder with the format \texttt{710-YYYYMMDD-G\#} (i.e., 710 followed by the date followed by the groups number) to the folder \texttt{11\--audiovisual\--recording\--assignment}. For example, if we recorded on March 10, 2020 and we were in group 9, then we would name the folder \texttt{710-20200310-G9}. Each file should have the same file naming convention.\\

\noindent The following files should be submitted in this folder:

\begin{enumerate}
  \item 50+ minute video file (e.g., \texttt{710-20200310-G9.mp4})
  \item 50+ minute audio file (e.g., \texttt{710-20200310-G9.wav})
  \item Setting photographs and/or diagram (e.g., \texttt{710-20200310-G9-A.jpg}, \texttt{710-20200310-G9-B.jpg}, \ldots)
  \item Event description (e.g., \texttt{710-20200310-G9.txt})
\end{enumerate}
\end{myenv}

\section{Selecting groups \& checking out equipment}
Once you have selected a group of two or three, input each person's name in the \texttt{audiovisual\--recording\--groups\--equipment\--checkout} spreadsheet file in the \texttt{11\--audiovisual\--recording\--assignment} on Google Drive.\\

\noindent Because the department currently has very few video recorders, it is my preference that you make the recordings in a room in Moore Hall. I realize that this may not be ideal for recording natural conversation. Once your group has decided on a time to record, please fill out the \texttt{audiovisual\--recording\--groups\--equipment\--checkout} spreadsheet file in the \texttt{11\--audiovisual\--recording\--assignment} on Google Drive.

%\begin{myenv}{Optional audio only}
%\noindent 
%
%\end{myenv}



\end{document}