\documentclass{article}

\usepackage[margin=1in]{geometry}
\usepackage{mdframed}
\newenvironment{myenv}[1]
  {\mdfsetup{
    frametitle={\colorbox{white}{\space#1\space}},
    innertopmargin=10pt,
    frametitleaboveskip=-\ht\strutbox,
    frametitlealignment=\center
    }
  \begin{mdframed}%!TEX encoding = UTF-8 Unicode
  }
  {\end{mdframed}}

\usepackage{fancyhdr}
\pagestyle{fancy}
\fancyhf{}
\rhead{2/16/2017}
\lhead{LING710}
\chead{}
\rfoot{Page \thepage}

%\pagestyle{firstpage}
%\fancyhf{}
% \textsc{Department of Linguistics}\vspace{0.75in}}}

\usepackage{enumitem}

\title{Syncing Audio \& Video with Adobe Premier Pro (2017)}
\date{}
%\pagenumbering{gobble}
\begin{document}\thispagestyle{empty}
%\maketitle
\begin{center}
{\Large\textbf{Syncing Audio \& Video with Adobe Premier Pro (2017)}}
\end{center}
%
%\begin{myenv}{Objectives}
%
%\end{myenv}

\section*{Creating an Adobe Premiere Pro file \texttt{.prproj} file}
\begin{enumerate}
  \item Open Adobe Premiere Pro (2017).
  \item Click \texttt{New project\ldots} on the pop-up start page.
  \item Name the project (e.g., FM2-20161007-G5) and select a directory to save the project (e.g., in the same folder as the original video files FM2-20161007-G5-O).
\end{enumerate}

\section*{Importing video and sound files into Adobe Premiere Pro}
\begin{enumerate}
  \item Click the `project pane' (lower left corner) that reads \texttt{Import media to start}. 
  \item Navigate to the \texttt{00000.mts} file and select it. 
    \begin{itemize}
      \item \texttt{\ldots/FM2-20161007-G5-O/PRIVATE/AVCHD/BDMV/STREAM/00000.mts}
    \end{itemize}
  \item Click the `project pane' again and navigate to the \texttt{FM2-20161007-G5.wav} file and select it.
  \item Both the video (.mts) and sound (.wav) file should appear in the `project pane'.
\end{enumerate}

\section*{Creating a `sequence' with audio and video files}
\begin{enumerate}
  \item Select the video (.mts) file and drag it to the `timeline pane'.
  \item You have now created a sequence file taking the same name as the .mts file.
  \begin{itemize}
    \item The sequence file contains slots for three video files (e.g., V1, V2, V3) and three audio files (e.g., A1, A2, A3). 
    \item The video (.mts) file takes up one video slot (V1) and one audio slot (A1)
  \end{itemize}
  \item Select the audio (.mts) file and drag it to the A2 slot in the `timeline pane'. 
  \begin{itemize}
    \item Note that the separate audio (.wav) and video (.mts) files do not align when it is played. Try playing the files in the `timeline pane', and you will hear two separate audio tracks.
  \end{itemize}
\end{enumerate}

\section*{Synchronizing audio and video in a sequence}
\begin{enumerate}
  \item Select both the audio and video files, so that they turn white.
  \item Right click on them, and select \texttt{Synchronize}.
  \item In the pop-up menu under \texttt{Synchronize Point}, select \texttt{Audio} and then \texttt{OK}.
  \begin{itemize}
    \item The audio and video files are no longer aligned from the beginning, but they are now perfectly in sync. 
  \end{itemize}
\end{enumerate}

\section*{Editing audio and video in a sequence}
\begin{enumerate}
  \item Select the \texttt{Razor Tool} just left of the `timeline pane'.
  \item Use the razor tool to `cut' the video and audio file in the same place. I advise that you make the cut just before I clap three times.
  \item Select the \texttt{Select Tool} just left of the `timeline pane'.
  \item Select the `extra piece' of audio and video that need to be discarded and press \texttt{delete}.
  \item Repeat the same process with the \texttt{Razor Tool} and \texttt{Select Tool} at the end of the recording (again, after I clap three times).
  \begin{itemize}
    \item Importantly, the audio and video segments are now \textit{exactly} the same length.
  \end{itemize}
  \item Select the audio and video files on the timeline and drag them to the beginning of the timeline.  
\end{enumerate}

\section*{Exporting the audio file}
\begin{enumerate}
  \item Mute the video (.mts) track by selecting both the eye icon so that it has a slash through it and the \texttt{M} icon immediately left of the track.
  \item Select the audio (A2, .wav) segment on the timeline.
  \item Click \texttt{File} $>$ \texttt{Export} $>$ \texttt{Media\ldots}.
  \item A pop-up menu \texttt{Export Settings} will appear.
  \item Under the format drop down menu, select \texttt{Waveform Audio}.
  \begin{itemize}
    \item The preset should be \texttt{WAV 48 kHz 16-bit}.
  \end{itemize}
  \item Click on the \texttt{Output Name} and select a location (e.g., in the directory FM2-20161007-G5) and name the file (e.g., FM2-20161007-G5.wav).
  \item Click \texttt{Export} at the bottom.
\end{enumerate}

\section*{Exporting the video file}
\begin{enumerate}
  \item Mute the audio (.wav) track by selecting the \texttt{M} immediately left of the track.
  \item Select the video segment on the timeline.
  \item Click \texttt{File} $>$ \texttt{Export} $>$ \texttt{Media\ldots}.
  \item A pop-up menu \texttt{Export Settings} will appear.
  \item Under the format drop down menu, select \texttt{H.264}.
  \begin{itemize}
    \item The preset should be \texttt{Match Source - High bitrate}.
  \end{itemize}
  \item Click on the \texttt{Output Name} and select a location (e.g., in the directory FM2-20161007-G5) and name the file (e.g., FM2-20161007-G5.mp4).
  \item Click \texttt{Export} at the bottom.
\end{enumerate}

\end{document}